\documentclass{report}
\usepackage[utf8]{inputenc} 
\usepackage[russian]{babel} 
\usepackage{enumerate}
\usepackage{fullpage}
\usepackage{indentfirst} 
\usepackage{amsfonts}
\renewcommand{\baselinestretch}{2}
\usepackage{pst-plot}
\usepackage{wrapfig}
\usepackage{xfrac}
\usepackage{listings}
\usepackage{color}
\usepackage{xcolor}
\usepackage{cmap}
\usepackage{spverbatim}
\usepackage{cmap}
\usepackage{tabu}
\usepackage{longtable}
\usepackage{caption}
\usepackage{multicol}
\usepackage{array}
\usepackage{multirow}
\usepackage{fancyvrb}

\usepackage[left=2cm,right=2cm,
    top=2cm,bottom=2cm,bindingoffset=0cm]{geometry}


\definecolor{string}{HTML}{B40000} % цвет строк в коде
\definecolor{comment}{HTML}{008000} % цвет комментариев в коде
\definecolor{keyword}{HTML}{1A00FF} % цвет ключевых слов в коде
\definecolor{morecomment}{HTML}{8000FF} % цвет include и других элементов в коде
%\definecolor{сaptiontextm}{HTML}{FFFFFF} % цвет текста заголовка в коде
%\definecolor{сaptionbk}{HTML}{999999} % цвет фона заголовка в коде
\definecolor{bk}{HTML}{FFFFFF} % цвет фона в коде
\definecolor{frame}{HTML}{999999} % цвет рамки в коде
\definecolor{brackets}{HTML}{B40000} % цвет скобок в коде


 % Настройки отображения кода
\lstset{
language=C++, % Язык кода по умолчанию
 % Цвета
keywordstyle=\color{keyword}\ttfamily\bfseries,
stringstyle=\color{string}\ttfamily,
commentstyle=\color{comment}\ttfamily\itshape,
morecomment=[l][\color{morecomment}]{\#},
 % Настройки отображения.....
breaklines=true, % Перенос длинных строк
basicstyle=\ttfamily\footnotesize, % Шрифт для отображения кода
backgroundcolor=\color{bk}, % Цвет фона кода
frame=lrb,xleftmargin=\fboxsep,xrightmargin=-\fboxsep, % Рамка, подогнанная к заголовку
rulecolor=\color{frame}, % Цвет рамки
tabsize=3, % Размер табуляции в пробелах
 % Настройка отображения номеров строк. Если не нужно, то удалите весь блок
numbers=left, % Слева отображаются номера строк
stepnumber=1, % Каждую строку нумеровать
numbersep=5pt, % Отступ от кода.
numberstyle=\small\color{black}, % Стиль написания номеров строк
}

\usepackage{caption}
\DeclareCaptionFont{white}{\color{white}}
\DeclareCaptionFormat{listing}{\parbox{\linewidth}{\colorbox{gray}{\parbox{\linewidth}{#3}}\vskip-4pt}}
\captionsetup[lstlisting]{format=listing,labelfont=white,textfont=white}


\author{{\huge Мингалёв Олег}\\
Московский государственный университет им. М. В. Ломоносова,\\
Факультет вычислительной математики и кибернетики,\\
101 группа\\
\texttt{oleg@mingalev.net}}
\title{Вычисление корней уравнений и определённых интегралов}
\date{23 февраля 2014 г.}

\begin{document}
\maketitle
%\addcontentsline{toc}{chapter}{Annexes}
\tableofcontents
\chapter{Постановка задачи}
\section{Общая задача}

С заданной точностью $ \varepsilon $ вычислить площадь плоской фигуры, ограниченной тремя кривыми, уравнения которых определяются функциями $ y = f_1(x) $, $ y = f_2(x) $ и $ y = f_3(x) $.

При решении задачи необходимо:
\begin{itemize}
    \item С некоторой точностью $ \varepsilon{}_1 $ вычислить абсциссы точек пересечения кривых.
    \item Представить площадь заданной фигуры как алгебраическую сумму определённых интегралов и вычислить эти интегралы с некоторой точностью $ \varepsilon{}_2 $.
\end{itemize}

\section{Исследуемые функции}
\[ f_1(x) = 0.6x + 3, \qquad
f_2(x) = (x-2)^3 - 1, \qquad
f_3(x) = 3/x \]
\[ \varepsilon = 10^{-3} \]

\chapter{Используемые численные методы}
\section{Решение уравнений: Метод касательных}

Основная идея метода заключается в следующем: задаётся начальное приближение вблизи предположительного корня, после чего строится касательная к исследуемой функции в точке приближения, для которой находится пересечение с осью абсцисс. Эта точка и берётся в качестве следующего приближения. И так далее, пока не будет достигнута необходимая точность.

Пусть $ f(x): \lbrack a, b\rbrack \rightarrow \mathbb{R} $ "--- определённая на отрезке $ \lbrack a, b \rbrack $ и дифференцируемая на нём вещественнозначная функция. Тогда формула итеративного исчисления приближений может быть выведена следующим образом:

\[ f'(x_n) = \tan \alpha = \frac{\Delta{}y}{\Delta{}x} = \frac{f(x_n) - 0}{x_n - x_{n+1}} = \frac{-f(x_n)}{x_{n+1} - x_n} \]

где $ \alpha $ "--- угол наклона касательной в точке $ x_n $.

Следовательно искомое выражение для $ x_{n+1} $ имеет вид:

\[ x_{n+1} = x_n - \frac{f(x_n)}{f'(x_n)} \]

Достаточным условием сходимости метода на отрезке $ \lbrack a, b \rbrack $ является выполнение следующих условий:

\begin{enumerate}
    \item Функция имеет разные знаки на концах отрезка;
    \item Первая и вторая производные функции на отрезке не меняют свой знак.
\end{enumerate}

\section{Вычисление определённого интеграла: Формула трапеций}

Метод трапеций — метод численного интегрирования функции одной переменной, заключающийся в замене на каждом элементарном отрезке подынтегральной функции на многочлен первой степени, то есть линейную функцию. Площадь под графиком функции аппроксимируется прямоугольными трапециями.

Если отрезок $ \lbrack a, b \rbrack $ является элементарным и не подвергается дальнейшему разбиению, значение интеграла можно найти по формуле

\[ \int^b_a f(x)\,dx = \frac{ f(a) + f(b) }{2} (b - a) + E(f), \qquad E(f) = - \frac{f''(\xi)}{12} \left( b-a \right)^3 \]

Если отрезок $ \lbrack a, b \rbrack $ разбивается узлами интегрирования и на каждом из элементарных отрезков применяется формула трапеций, то суммирование даст составную формулу трапеций

\[ \int^b_a f(x)\,dx \approx \sum_{i=0}^{n-1} \frac{ f(x_i) + f(x_{i+1}) }{2} (x_{i+1} - x_{i}) = \frac{f(a)}{2} (x_1 - a) + \sum_{i=1}^{n-1} f(x_i) \frac{ x_{i+1} - x_{i-1} } {2} + \frac{f(b)}{2} (b - x_{n-1}) \]

И, наконец, в случае равномерной сетки:

\[ \int^b_a f(x)\,dx = h \left( \frac{f_0 + f_n}{2} + \sum_{i=1}^{n-1} f_i \right) + E_n(f), \qquad E_n(f) = - \frac{f''(\xi)}{12} (b - a) h^2 \]

\chapter{Аналитическая часть}

\section{Приближённый поиск корней}

\psset{
    axesstyle=frame,ysubticks=8,
    xsubticks=4,subticksize=0.5,
    subtickcolor=black,
}

\begin{wrapfigure}[3]{r}{0.5\textwidth}
\begin{psgraph}{->}(0,0)(5,6){6cm}{6cm}
\rput(0.25,5.5){\psline[linecolor=red]
    (0,0)(0.75cm,0)}
\rput[l](1.0,5.5){$0.6*x+3$}
\rput(0.25,5){\psline[linecolor=blue]
    (0,0)(0.75cm,0)}
\rput[l](1,5){$(x-2)^3-1$}
\rput(0.25,4.5){\psline[linecolor=green]
    (0,0)(0.75cm,0)}
\rput[l](1,4.5){$3/x$}
\psplot[algebraic,linecolor=red,linewidth=1pt]{0.5}{4}{0.6*x+3}
\psplot[algebraic,linecolor=blue,linewidth=1pt]{3.1}{3.9}{(x-2)*(x-2)*(x-2)-1}
\psplot[algebraic,linecolor=green,linewidth=1pt]{0.7}{4}{3/x}
\end{psgraph}
\end{wrapfigure}

$$
\left.
    \begin{array}{lcr}
        \left\{
        \begin{array}{lcl}
            y & = & 0.6*x+3 \\
            y & = & (x-2)^3 - 1
        \end{array}
        \right.& \rightarrow & y \in (3, 4)
    \\
    \left\{
    \begin{array}{lcl}
        y & = & 0.6*x+3 \\
        y & = & 3/x
    \end{array}
    \right. & \rightarrow & y \in (\sfrac{1}{2}, 1)
    \\
    \left\{
    \begin{array}{lcl}
        y & = & (x-2)^3 - 1 \\
        y & = & 3/x
    \end{array}
    \right. & \rightarrow & y \in (3, 4)
\end{array}
\right.
$$

\clearpage
\section{Обоснование применимости метода поиска корней}

Для функций $ f_{12} = f_1 - f_2 $, $ f_{13} = f_1 - f_3 $ и $ f_{23} = f_2 - f_3 $ покажем, что на концах выбранных ранее отрезков функции имеет разные знаки, а первые и вторые производные на этих отрезках не меняют свой знак.

\[ f_{12}(3) \approx 4.8 > 0 > f_{12}(4) \approx -1.6 \]
\[ f_{12}' = (0.6) - (3(x-2)^3) = -3x^2 +12x -11.4 < 0 \qquad \forall x \in (3, 4) \]
\[ f_{12}'' = (0) - (6x - 12) = 12 - 6x < 0 \qquad \forall x \in (3, 4) \]

\[ f_{13}(\sfrac{1}{2}) \approx -2.7 < 0 < f_{13}(1) \approx 0.6 \]
\[ f_{13}' = (0.6) - (-\frac{3}{x^2}) = 0.6 + \frac{3}{x^2} > 0 \qquad \forall x \in (\sfrac{1}{2}, 1) \]
\[ f_{13}'' = (0) - (\frac{6}{x^3}) = -\frac{6}{x^3} < 0 \qquad \forall x \in (\sfrac{1}{2}, 1) \]

\[ f_{23}(3) \approx -1 < 0 < f_{23}(4) \approx 6.25 \]
\[ f_{23}' = (3(x-2)^2) - (-\frac{3}{x^2}) = 3x^2 + \frac{3}{x^2} - 12x + 12 > 0 \qquad \forall x \in (3, 4) \]
\[ f_{23}'' = (6x - 12) - (\frac{6}{x_3}) = -\frac{6}{x^3} + 6x - 12 > 0 \qquad \forall x \in (3, 4) \]

\chapter{Тестирование}
\begin{Verbatim}[frame=single,baselinestretch=0.5]
[shhdup@shhdup-think lab01]$ ./lab01 -t
lab01 v0.1.0, Mingalev Oleg 2014

Use lab01 -h to see help page

Variant: 6-3-2

Functions:
 0.6x+3
 (x-2)^3-1
 3/x

Root approximation method: Newton's method
Definite integral approximation method: Trapezoidal rule
========================================================

=== Self-testing ===
== Testing root() ==
1. [x^3-2x] x = -1.414214 f(x) = -0.000000 [OK]
2. [1-1/x] x = 1.000000 f(x) = -0.000000 [OK]
3. [-2x-2^(-x)] x = -1.000000 f(x) = -0.000000 [OK]
4. [x^3-x] x = 1.000001 f(x) = 0.000002 [OK]
== Testing integral() ==
1. [x^2] [0.000000;9.000000] I=243.000000 true=243.000000 [OK]
2. [1/x] [1.000000;2.718281] I=1.000000 true=1.000000 [OK]
3. [sqrt(1-x^2)] [-1.000000;1.000000] I=1.570796 true=1.570796 [OK]
=!= All tests are OK =!=
\end{Verbatim}

\chapter{Результаты счёта}
\begin{Verbatim}[frame=single,baselinestretch=0.5]
[shhdup@shhdup-think lab01]$ ./lab01 -r
lab01 v0.1.0, Mingalev Oleg 2014

Use lab01 -h to see help page

Variant: 6-3-2

Functions:
 0.6x+3
 (x-2)^3-1
 3/x

Root approximation method: Newton's method
Definite integral approximation method: Trapezoidal rule
========================================================

(1, 2): x = 3.847760, f1(3.847760) = 5.308656, f2(3.847760) = 5.308656
(1, 3): x = 0.854102, f1(0.854102) = 3.512461, f3(0.854102) = 3.512462
(2, 3): x = 3.243929, f2(3.243929) = 0.924807, f3(3.243929) = 0.924804
Area: 7.488425
\end{Verbatim}

\appendix
\chapter{Исходный код}

\lstinputlisting[caption=Makefile]{../Makefile}
\lstinputlisting[language=C, caption=lab01.h]{../lab01.h}
\lstinputlisting[language=C, caption=lab01.c]{../lab01.c}
\lstinputlisting[language=C, caption=variant.h]{../variant.h}
\lstinputlisting[language=C, caption=variant.c]{../variant.c}
\lstinputlisting[language=C, caption=test.h]{../test.h}
\lstinputlisting[language=C, caption=test.c]{../test.c}

\end{document}

